\section{Affine Space and Algebraic Sets}
\begin{prob} \label{n=1:代数的ならば有限}
  $\A^{1}(k)$の代数的部分集合は有限部分集合であることを示せ.
\end{prob}
\begin{ans}
  $X \subset \A^1(k)$を空でない代数的部分集合であるとする.
  $X = V( S )$となる$k[X]$の部分集合$S \ne \emptyset$が存在する.
  $f \in S$を取る.
  $f = \sum_{i=0}^n a_i X^i, a_i \in k$
  と書ける.
  $f$の根は高々$n$個しかない.
  よって$X = V(S) \subseteq V(f)$であるから
  $X$は有限集合である.
\end{ans}

\begin{prob}
  $k$が有限体ならば$\A^n(k)$の任意の部分集合は代数的であることを示せ.
\end{prob}
\begin{ans}
  $k$を位数$q$の有限体とする.
  任意の$p = (a^{(p)}_1, \dots , a^{(p)}_n ) \in \A^n(k)$について,
  $V( X_1 - a^{(p)}_1, \dots , X_n - a^{(p)}_n ) = \{ p \} \subset A^n(k)$
  である.
  $\A^n(k)$の任意の部分集合$X$は有限集合である.
  $X = \{ p_1, \dots , p_m \}$とおいて,
  $X = V(X_1 - a^{(p_1)}_1, \dots , X_n - a^{(p_1)}_n ) \cup \cdots \cup V(X_1 - a^{(p_m)}_1, \dots , X_n - a^{(p_m)}_n ) $
  であり,有限個の代数的集合の和は代数的集合であるので,$X$は代数的集合である.
\end{ans}

\begin{prob}
  代数的集合の可算個の族であって,和が代数的でないような例を挙げよ.
\end{prob}
\begin{ans}
  $\A^1(\R)$の可算個の集合たち$\{1\}, \{2\}, \dots $を考える.
  これらはそれぞれ有限集合なので代数的集合である.
  一方これらの和$\{1,2,\dots\}$は無限集合である.
  したがって問題 \ref{n=1:代数的ならば有限} より,これは代数的集合ではない.
\end{ans}

\begin{prob}
  次の集合が代数的であることを示せ.
  \begin{itemize}
    \item[(a)] $\{ (t,t^2,t^3 \in \A^3(k) \mid t \in k \}$
    \item[(b)] $\{ (\cos t, \sin t) \in \A^2(\R) \mid t \in \R \} $
    \item[(c)] $\A^2(\R)$の部分集合であって各点の極座標$(r,\theta)$が$r=\sin \theta$をみたすもの
  \end{itemize}
\end{prob}
\begin{ans}
  (a).
  $X^2-Y, X^3-Z \in k[X,Y,Z]$を考えると,
  $V(X^2-Y,X^3-Z) = \{ (t,t^2,t^3 \in \A^3(k) \mid t \in k \}$となるので
  これは代数的である.
  
  (b).
  $V(X^2 + Y^2 - 1) = \{ (\cos t, \sin t) \in \A^2(\R) \mid t \in \R \} $である.
  よってこれは代数的である.

  (c).
  $X = r \cos \theta, Y = r \sin \theta$なので
  $ r = \sin \theta
  \Leftrightarrow r^2 = r \sin \theta
  \Leftrightarrow X^2 + Y^2 = Y
  \Leftrightarrow X^2 + Y^2 - Y = 0 $
  である.
  したがって,$V(X^2 - Y^2 - Y) = \{ p \in \A^2(\R) \mid r = \sin \theta \} $
  であるから,代数的である.
\end{ans}
