\section{Affine Space and Algebraic Sets}
\begin{prob}
  $\A^{1}(k)$の代数的部分集合は有限部分集合であることを示せ.
\end{prob}
\begin{ans}
  $X \subset \A^1(k)$を空でない代数的部分集合であるとする.
  $X = V( S )$となる$k[X]$の部分集合$S \ne \emptyset$が存在する.
  $f \in S$を取る.
  $f = \sum_{i=0}^n a_i X^i, a_i \in k$
  と書ける.
  $f$の根は高々$n$個しかない.
  よって$X = V(S) \subseteq V(f)$であるから
  $X$は有限集合である.
\end{ans}

\begin{prob}
  $k$が有限体ならば$\A^n(k)$の任意の部分集合は代数的であることを示せ.
\end{prob}
\begin{ans}
  $k$を位数$q$の有限体とする.
  任意の$p = (a^{(p)}_1, \dots , a^{(p)}_n ) \in \A^n(k)$について,
  $V( X_1 - a^{(p)}_1, \dots , X_n - a^{(p)}_n ) = \{ p \} \subset \in A^n(k)$
  である.
  $\A^n(k)$の任意の部分集合$X$は有限集合である.
  $X = \{ p_1, \dots , p_m \}$とおいて,
  $X = V(X_1 - a^{(p_1)}_1, \dots , X_n - a^{(p_1)}_n ) \cup \cdots \cup V(X_1 - a^{(p_m)}_1, \dots , X_n - a^{(p_m)}_n ) $
  であり,有限個の代数的集合の和は代数的集合であるので,$X$は代数的集合である.
\end{ans}
