\section{Affine Space and Algebraic Sets}
\begin{prob} \label{n=1:代数的ならば有限}
  $\A^{1}(k)$の代数的部分集合は有限部分集合であることを示せ.
\end{prob}
\begin{ans}
  $X \subset \A^1(k)$を空でない代数的部分集合であるとする.
  $X = V( S )$となる$k[X]$の部分集合$S \ne \emptyset$が存在する.
  $f \in S$を取る.
  $f = \sum_{i=0}^n a_i X^i, a_i \in k$
  と書ける.
  $f$の根は高々$n$個しかない.
  よって$X = V(S) \subseteq V(f)$であるから
  $X$は有限集合である.
\end{ans}

\begin{prob}
  $k$が有限体ならば$\A^n(k)$の任意の部分集合は代数的であることを示せ.
\end{prob}
\begin{ans}
  $k$を位数$q$の有限体とする.
  任意の$p = (a^{(p)}_1, \dots , a^{(p)}_n ) \in \A^n(k)$について,
  $V( X_1 - a^{(p)}_1, \dots , X_n - a^{(p)}_n ) = \{ p \} \subset A^n(k)$
  である.
  $\A^n(k)$の任意の部分集合$X$は有限集合である.
  $X = \{ p_1, \dots , p_m \}$とおいて,
  $X = V(X_1 - a^{(p_1)}_1, \dots , X_n - a^{(p_1)}_n ) \cup \cdots \cup V(X_1 - a^{(p_m)}_1, \dots , X_n - a^{(p_m)}_n ) $
  であり,有限個の代数的集合の和は代数的集合であるので,$X$は代数的集合である.
\end{ans}

\begin{prob}
  代数的集合の可算個の族であって,和が代数的でないような例を挙げよ.
\end{prob}
\begin{ans}
  $\A^1(\R)$の可算個の集合たち$\{1\}, \{2\}, \dots $を考える.
  これらはそれぞれ有限集合なので代数的集合である.
  一方これらの和$\{1,2,\dots\}$は無限集合である.
  したがって問題 \ref{n=1:代数的ならば有限} より,これは代数的集合ではない.
\end{ans}

\begin{prob}
  次の集合が代数的であることを示せ.
  \begin{itemize}
    \item[(a)] $\{ (t,t^2,t^3 \in \A^3(k) \mid t \in k \}$
    \item[(b)] $\{ (\cos t, \sin t) \in \A^2(\R) \mid t \in \R \} $
    \item[(c)] $\A^2(\R)$の部分集合であって各点の極座標$(r,\theta)$が$r=\sin \theta$をみたすもの
  \end{itemize}
\end{prob}
\begin{ans}
  (a).
  $X^2-Y, X^3-Z \in k[X,Y,Z]$を考えると,
  $V(X^2-Y,X^3-Z) = \{ (t,t^2,t^3 \in \A^3(k) \mid t \in k \}$となるので
  これは代数的である.
  
  (b).
  $V(X^2 + Y^2 - 1) = \{ (\cos t, \sin t) \in \A^2(\R) \mid t \in \R \} $である.
  よってこれは代数的である.

  (c).
  $X = r \cos \theta, Y = r \sin \theta$なので
  $ r = \sin \theta
  \Leftrightarrow r^2 = r \sin \theta
  \Leftrightarrow X^2 + Y^2 = Y
  \Leftrightarrow X^2 + Y^2 - Y = 0 $
  である.
  したがって,$V(X^2 - Y^2 - Y) = \{ p \in \A^2(\R) \mid r = \sin \theta \} $
  であるから,代数的である.
\end{ans}

\begin{prob}
  $C$をアフィン平面曲線とし,
  $L$を$\A^2(k)$の直線とする.
  $L \not \subset C$である.
  次数$n$の多項式$F$で
  $C = V(F), \ F \in k[X,Y]$と書けるとする.
  $L \cap C$は高々$n$点の有限集合であることを示せ.
\end{prob}
\begin{ans}
  $L = V(Y-(aX+b))$と書けるとしてよい.
  このとき,
  $C \cap L = V( F, Y-(aX+b) ) = V( F(X,Y-(aX+b) )$である.
  $F(X,aX+b) \in k[X]$は$F \not \subset C$なので$0$ではない.
  $F(X,aX+b)$の次数は高々$n$であるので
  根も高々$n$個である.
  したがって$L\cap C$は高々$n$点の有限集合である.
\end{ans}

\begin{prob}
  次の集合が代数的でないことを示せ.
  \begin{itemize}
    \item[(a)] $\{(x,y) \in \A^2(\R) \mid y = \sin x\}$
    \item[(b)] $\{( z,w ) \in \A^2(\C) \mid |z|^2 + |w|^2 = 1 \}$
    \item[(c)] $\{ (\cos t, \sin t, t) \in \A^3(\R) \mid t \in \R\}$
  \end{itemize}
\end{prob}
\begin{ans}
  (a).
  $A := \{(x,y) \in \A^2(\R) \mid y = \sin x\}$が代数的集合であると仮定する.
  多項式の集合$S \subset \R[X,Y]$が存在して$A = V(S)$と書ける.
  環準同型$\phi : \R[X,Y] \to \R[X]$を
  $\phi(a) = a, a \in \R$,$\phi(X) = X$,$\phi(Y) = 0$を満たすものとして定める.
  このとき$\phi(S) = \{\phi(F) \mid F \in S\}$とおくと,
  $V(\phi(S)) \subset \A^1(\R)$は無限個の点$\{ n\pi \mid n \in \Z \}$を含む代数的集合となる.
  しかしこれは問題\ref{n=1:代数的ならば有限}の結果に反する.
  したがって$A$は代数的でない.
  
  (b).
  $B := \{ (z,w) \in \A^2(\C) \mid |z|^2 + |w|^2 = 1 \}$が代数的であると仮定する.
  多項式の集合$T \subset \C[Z,W]$が存在して$B = V(S)$と書ける.
  環準同型$\psi : \C[Z,W] \to \C[Z]$を
  $\psi(a) = a, a \in \C$,$\psi(Z) = Z$,$\psi(W) = 0$を満たすものとして定める.
  このとき$\psi(T) = \{\psi(F) \mid F \in T\}$とおくと,
  $V(\psi(T)) \subset \A^1(\C)$は無限個の点$\{ z \in \C \mid |z|^2 = 1 \}$を含む代数的集合となる.
  しかしこれは問題\ref{n=1:代数的ならば有限}の結果に反する.
  したがって$B$は代数的でない.

  (c).
  $C := \{ (\cos t, \sin t, t) \in \A^3(\R) \mid t \in \R\}$が代数的であると仮定する.
  多項式の集合$U \subset \R[X,Y,Z]$が存在して$B = V(U)$と書ける.
  環準同型$\theta : \R[X,Y,Z] \to \R[Z]$を
  $\theta(a) = a, a \in \R$,$\theta(X) = 1$,$\theta(Y) = 0$,$\theta(Z) = Z$を満たすものとして定める.
  このとき$\theta(U) = \{\theta(F) \mid F \in U\}$とおくと,
  $V(\theta(U)) \subset \A^1(\R)$は無限個の点$\{ 2n\pi \mid n \in \Z \}$を含む代数的集合となる.
  しかしこれは問題\ref{n=1:代数的ならば有限}の結果に反する.
  したがって$C$は代数的でない.
\end{ans}

\begin{prob}
  $k$を代数的閉体,$F\in k[X_1,\dots,X_n]$を定数でない多項式とする.
  $n\ge 1$ならば$\A^n(k) \setminus V(F)$が無限集合であり,
  $n\ge 2$ならば$V(F)$が無限集合であることを示せ.
  そして任意の真の代数的集合の補集合は無限集合であることを結論付けよ.
\end{prob}
\begin{ans}
  $k$が代数的閉体なので問題\ref{代数的閉体は無限集合}より
  $k$は無限集合であり,$n\ge 1$のとき$\A^n(k)$は無限集合である.
  
  $n\ge 1$とする.
  もし$\A^n(k) \setminus V(F)$が有限ならば
  $A^n(k)$が無限集合であることから
  $V(F)$は無限集合である.
  すると$F(X_1,\dots, X_n) \in k[X_1,\dots,X_n]$に
  任意の$a_1,\dots,a_{n-1} \in k$を代入して得られる多項式
  $\bar{F}(X_n) = F(a_1,\dots ,a_{n-1}, X_n) \in k[X_n]$
  は高々有限個の$k$の元を除いて$\bar{F}(a) = 0 \ a\in k$となる.
  つまり$\bar{F}$は無限個の根を持たなければならないが,これは不可能である.
  よって$\A^n(k) \setminus V(F)$は無限集合である.

  $n\ge 2$ならば$V(F)$が無限集合であることを数学的帰納法で証明する.
  $n = 2$とする.
  $F(X,Y) \in k[X,Y]$は$X$について定数でないとしてよい.
  任意の$a \in k$に対して
  $F(X,a) \in k[X]$となり,これは定数ではない.
  $k$が代数的閉体であることから
  $F(X,a)$はすくなくとも一つ根をもつ.これを$r_a \in k$をする.
  いま$a\in k$は任意であったから
  $ \{ (r_a, a) \mid a \in k \} \subset V(F) $である.
  $k$は無限集合なので$V(F)$が無限集合であることが分かる.
  
  $n-1$以下のときは主張が成り立つとして$n$のとき,
  $F(X_1, \dots, X_n) \in k[X_1,\dots,X_n]$
  は$X_1$について定数でないとしてよい.
  $a \in k$を任意にとり$X_n$に代入して得られる多項式は
  $F(X_1, \dots, X_{n-1}, a) \in k[X_1,\dots,X_{n-1}]$
  となり,帰納法の仮定から$V(F) \subset \A^{n-1}(k)$は
  無限集合となる.
  したがって$V(F) \subset \A^n(k)$は無限集合である.

  $A = V(S)$を代数的集合であるとする.
  $S = \{ F_{\lambda} \}_{\lambda\in\Lambda}$とおく.
  $\A^n(k) \setminus A$が無限集合であることを示す.
  \[ \A^n(k) \setminus A
  = \A^n(k) \setminus \bigcap_{\lambda\in\Lambda} V(F_{\lambda})
  = \bigcup_{\lambda\in\Lambda} \left( \A^n(k) \setminus V(F_{\lambda}) \right) \]
  となる.
  上で示したことから$\A^n(k) \setminus V(F_{\lambda}$は無限集合なので
  $ \A^n(k) \setminus A$は無限集合である.
\end{ans}

\begin{prob}
  $V \subset \A^{n}(k), W\subset \A^m(k)$を代数的集合とする.
  このとき,
  $V\times W = \{ (a_1, \dots, a_n, b_1, \dots b_m) \in \A^{n+m}(k) \mid (a_1, \dots a_n) \in V, (b_1, \dots, b_m) \in W \}$
  が代数的集合であることを示せ.
  これを$V$と$W$の積という.
\end{prob}
\begin{ans}
  $V,W$は代数的集合なので
  $S\subset k[X_1,\dots,X_n], T \subset k[Y_1, \dots, Y_m]$が存在して
  $V = V( S ), W = V( T )$となる.
  $S,T$に属する多項式を$k[X_1,\dots,X_n,Y_1,\dots,Y_m]$の元と思うと,
  $V(S \cup T) = V \times W$であることが分かる.
\end{ans}
