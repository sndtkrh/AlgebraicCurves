\section{Irreducible Components of an Algebraic Set}

\begin{prob}
  ネーター環のイデアルの族$\mathcal{I}$であって
  $\mathcal{I}$の極大元が極大イデアルではないようなものの例を挙げよ.
\end{prob}
\begin{ans}
  $\Z$はネーター環である.
  $c \in \Z$を合成数として
  $\mathcal{I} = (c)$とすると,
  極大元は$(c)$であるが,これは極大イデアルではない.
\end{ans}

\begin{prob}
  ネーター環の全ての真のイデアルは極大イデアルに含まれていることを示せ.
\end{prob}
\begin{ans}
  $R$をネーター環とする.
  任意にイデアル$I \subset R$をとる.
  $ \mathcal{I} := \{ J \subsetneq R \mid J:\text{イデアル}, I \subset J \} $
  とおくと$R$がネーター環であることから
  極大元$I_0 \in \mathcal{I}$が存在する.
  これは$R$の極大イデアルであり,$I$を含んでいる.
\end{ans}

\begin{prob}
  (a) $V(Y-X^2) \subset \A^2(\C)$が既約であることを示せ.
  実際,$I(V(Y-X^2)) = (Y-X^2)$である.
  (b) $V(Y^4-X^2, Y^4 - X^2Y^2 + XY^2 - X^3) \subset \A^2(\C)$を既約成分に分解せよ.
\end{prob}
\begin{ans}
  (a).
  $V(Y-X^2) = \{ (x,y) \in \A^2(\C) \mid y-x^2 = 0 \}$
  なので
  $I(V(Y-X^2)) = (Y-X^2)$
  である.
  $Y-X^2 \in \C[X,Y]$は$Y$についての一次式なので既約元である.
  $\C[X,Y]$はUFDだから既約元は素元であるので
  $(Y-X^2) \subset \C[X,Y]$は素イデアルである.
  したがって$V(Y-X^2)$は既約である.
  
  (b).
  $V(Y^4-X^2, Y^4 - X^2Y^2 + XY^2 - X^3) = V( (Y^2-X)(Y^2+X), (Y+X)(Y-X)(Y^2+X) )
  = V( Y^2-X, Y^2-X^2 ) \cup V( Y^2+X ) $
  である.
  $V(Y^2 + X)$は既約である.
  また,
  $ V( Y^2-X, Y^2-X^2 ) = \{ (x,y) \in \A^2(\C) \mid y^2-x=0, y^2-x^2=0 \}
  = \{ (0,0), (1,1) \} = V(X,Y) \cup V(X-1,Y-1) $
  である.
  $I(V(X,Y)) = (X,Y), I(V(X-1,Y-1)) = (X-1,Y-1)$
  はそれぞれ素イデアルである.
  $(X-1,Y-1)$が素イデアルであることを示す.
  $ G, H \not \in (X-1,Y-1) $ならば
  \[ \begin{split}
    G &= \sum_{i} \lambda_{i} (X-1)^{s_i}(Y-1)^{t_i} + c, \ s_i,t_i \ge 1, c \in \C \\
    H &= \sum_{i} \mu_{i}     (X-1)^{u_i}(Y-1)^{v_i} + d, \ u_i,v_i \ge 1, d \in \C
  \end{split} \]
  と書けるので
  \[ GH = \sum_{i} \lambda_{xi} (X-1)^{p_i}(Y-1)^{q_i} + cd, \ p_i,q_i \ge 1 \]
  となる.
  したがって$GH \not \in (X-1,Y-1)$であるから$(X-1,Y-1)$は素イデアルである.
  $V(X,Y)$が素イデアルであることも同様に示される.
  以上から,
  $V(X,Y), V(X-1,Y-1) $は既約であり,
  $V(Y^4-X^2, Y^4 - X^2Y^2 + XY^2 - X^3) = V(X,Y) \cup V(X-1,Y-1) \cup V( Y^2+X ) $
  と既約成分に分解される.
\end{ans}
