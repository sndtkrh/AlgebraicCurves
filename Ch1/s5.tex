\section{Irreducible Components of an Algebraic Set}

\begin{prob}
  ネーター環のイデアルの族$\mathcal{I}$であって
  $\mathcal{I}$の極大元が極大イデアルではないようなものの例を挙げよ.
\end{prob}
\begin{ans}
  $\Z$はネーター環である.
  $c \in \Z$を合成数として
  $\mathcal{I} = (c)$とすると,
  極大元は$(c)$であるが,これは極大イデアルではない.
\end{ans}

\begin{prob}
  ネーター環の全ての真のイデアルは極大イデアルに含まれていることを示せ.
\end{prob}
\begin{ans}
  $R$をネーター環とする.
  任意にイデアル$I \subset R$をとる.
  $ \mathcal{I} := \{ J \subsetneq R \mid J:\text{イデアル}, I \subset J \} $
  とおくと$R$がネーター環であることから
  極大元$I_0 \in \mathcal{I}$が存在する.
  これは$R$の極大イデアルであり,$I$を含んでいる.
\end{ans}

\begin{prob}
  (a) $V(Y-X^2) \subset \A^2(\C)$が既約であることを示せ.
  実際,$I(V(Y-X^2)) = (Y-X^2)$である.
  (b) $V(Y^4-X^2, Y^4 - X^2Y^2 + XY^2 - X^3) \subset \A^2(\C)$を既約成分に分解せよ.
\end{prob}
\begin{ans}
  (a).
  $V(Y-X^2) = \{ (x,y) \in \A^2(\C) \mid y-x^2 = 0 \}$
  なので
  $I(V(Y-X^2)) = (Y-X^2)$
  である.
  $Y-X^2 \in \C[X,Y]$は$Y$についての一次式なので既約元である.
  $\C[X,Y]$はUFDだから既約元は素元であるので
  $(Y-X^2) \subset \C[X,Y]$は素イデアルである.
  したがって$V(Y-X^2)$は既約である.
  
  (b).
  $V(Y^4-X^2, Y^4 - X^2Y^2 + XY^2 - X^3) = V( (Y^2-X)(Y^2+X), (Y+X)(Y-X)(Y^2+X) )
  = V( Y^2-X, Y^2-X^2 ) \cup V( Y^2+X ) $
  である.
  $V(Y^2 + X)$は既約である.
  また,
  $ V( Y^2-X, Y^2-X^2 ) = \{ (x,y) \in \A^2(\C) \mid y^2-x=0, y^2-x^2=0 \}
  = \{ (0,0), (1,1) \} = V(X,Y) \cup V(X-1,Y-1) $
  である.
  $I(V(X,Y)) = (X,Y), I(V(X-1,Y-1)) = (X-1,Y-1)$
  はそれぞれ素イデアルである.
  $(X-1,Y-1)$が素イデアルであることを示す.
  $ G, H \not \in (X-1,Y-1) $ならば
  \[ \begin{split}
    G &= \sum_{i} \lambda_{i} (X-1)^{s_i}(Y-1)^{t_i} + c, \ s_i,t_i \ge 1, c \in \C \\
    H &= \sum_{i} \mu_{i}     (X-1)^{u_i}(Y-1)^{v_i} + d, \ u_i,v_i \ge 1, d \in \C
  \end{split} \]
  と書けるので
  \[ GH = \sum_{i} \lambda_{xi} (X-1)^{p_i}(Y-1)^{q_i} + cd, \ p_i,q_i \ge 1 \]
  となる.
  したがって$GH \not \in (X-1,Y-1)$であるから$(X-1,Y-1)$は素イデアルである.
  $V(X,Y)$が素イデアルであることも同様に示される.
  以上から,
  $V(X,Y), V(X-1,Y-1) $は既約であり,
  $V(Y^4-X^2, Y^4 - X^2Y^2 + XY^2 - X^3) = V(X,Y) \cup V(X-1,Y-1) \cup V( Y^2+X ) $
  と既約成分に分解される.
\end{ans}

\begin{prob}
  $F = Y^2 + X^2(X-1)^2 \in \R[X,Y] $は
  既約多項式であるが$V(F)$は可約であることを示せ.
\end{prob}
\begin{ans}
  まず,$F$が$\R[X,Y]$で既約であることを示す.
  $F$が可約であると仮定すると
  定数でない多項式$G,H \in \R[X,Y]$によって
  $F = GH$と表される.
  必要ならば$G,H$を入れ替えて
  $G$は$Y$について定数であるか$1$次式であるとしてよい.
  $G$が$Y$について定数であるなら
  $H$は$Y$について$2$次式であるから,
  $F = G(H_2Y^2 + H_1Y + H_0) = GH_2Y^2 + GH_1Y + GH_0 ,\ H_i \in \R[X] $
  となるが$Y^2$の係数が$X$の$1$次以上の多項式となっていて矛盾.
  また,$G$が$Y$について$1$次式のとき,
  $H$も$Y$について$1$次式である.
  よって
  $ F = GH = (G_1Y + G_0)(H_1Y + H_0)
  = G_1H_1 Y^2 + (G_0H_1 + G_1H_0) Y + G_0H_0, \ G_i, H_i \in \R[X] $
  であるから係数を比較して
  $G_1 H_1 = 1, G_0 H_1 = -G_1 H_0, G_0H_0 = X^2(X-1)^2 $
  でなければならない.これらより
  $ G_0 = c H_0 , \ c < 0$とおけて
  $c H_0^2 = X^2(X-1)^2 $となることが分かるが
  左辺の最高次の項の係数は負であり,右辺の最高次の項の係数は正である.
  これは矛盾である.
  したがって$F$は$\R[X,Y]$上既約である.
  
  次に$V(F)$が可約であることを示す.
  $I(V(F))$が素イデアルではないことを示せばよい.
  $I(V(F)) = I(V(Y^2 + X^2(X-1)^2))
  = I( \{ (x,y)\in\A^2(\R) \mid y^2 = -x^2(x-1)^2 \} )
  = I( \{ (0,0), (1,0) \} )
  = (X(X-1), Y) $
  となる.
  $X(X-1) \in I(V(F))$だが
  $X,(X-1) \not\in I(V(F)) $なので
  $I(V(F))$は素イデアルでない.
\end{ans}

\begin{prob}
  $V,W\subset \A^n(k)$を$V\subset W$である代数的集合とする.
  $V$の各既約成分は$W$のある既約成分に含まれることを示せ.
\end{prob}
\begin{ans}
  $V'$を$V$の既約成分とする.
  $W = W_1 \cup \cdots \cup W_m $と既約分解する.
  $V' \not \subset W_i, \forall i$であると仮定する.
  もしある$i$に対して$W_i \subset V'$なら
  $W = W_1 \cup \cdots \cup W_{i-1} \cup V' \cup W_{i+1} \cup \cdots \cup W_m $が,
  そうでないなら$W = V' \cup W_1 \cup \cdots \cup W_m $が$W$の既約分解となっている.
  しかしこれは既約分解の一意性に反する.
  ゆえにある$i$が存在して$V' \subset W_i $となっている.
\end{ans}

\begin{prob}
  $V = V_1 \cup \cdots \cup V_r $が代数的集合の既約成分への分解であるとき,
  $ V_i \not \subset \bigcup_{j\ne i} V_j$を示せ.
\end{prob}
\begin{ans}
  もしある$i=1$で
  $ V_1 \subset \bigcup_{j\ne 1} V_j$
  となっているならば
  $ V = V_2 \cup \cdots \cup V_r $も既約成分分解となっている.
  しかし,これは既約成分分解の一意性に反する.
\end{ans}

\begin{prob}
  $k$が無限体のとき,
  $\A^n(k)$は既約であることを示せ.
\end{prob}
\begin{ans}
  $k$が無限体なので
  問題 \ref{全ての点に対して0なら零多項式} より
  $\A^n(k)$のすべての点$a$に対して$F(a) = 0$となる多項式は$ F = 0 $のみだから
  $I( \A^n(k) ) = ( 0 ) \subset k[X_1,\dots X_n]$である.
  $(0)$は素イデアルであるので
  $\A^n(k)$は既約である.
\end{ans}
