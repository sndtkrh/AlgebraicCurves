\section{Algebraic Subsets of the Plane}

\begin{prob}
  $k=\R$とする.
  (a) $I(V(X^2+Y^2+1)) = (1)$を示せ.
  (b) $\A^2(\R)$の全ての代数的な部分集合は,
  ある$F\in\R[X,Y]$に対して$V(F)$と表されることを示せ.
\end{prob}
\begin{ans}
  (a).
  $V(X^2+Y^2+1) = \emptyset$なので
  $I(V(X^2+Y^2+1)) = I(\emptyset) = (1)$

  (b).
  $V\subset \A^2(\R)$を代数的集合とする.
  $V = \A^2(\R)$のときは$V = V(0)$なのでよい.
  $V \ne \A^2(\R)$とする.
  $V = V_1 \cup \cdots \cup V_r$と既約成分分解する.
  本文中 Corollary 2 より,
  $V_i$は一点か$V(F_i)$である.
  ここで$F_i \in \R[X,Y]$は既約多項式である.
  $V_i \ne \A^2(\R)$のとき,
  各$i$に対して,$V_i = \{ (a,b) \}$のときは
  $F_i = (X-a)^2 + (Y-b)^2 \in \R[X,Y]$とおく.
  すると$V_i = V(F_i)$である.
  したがって,
  $V = V(F_1) \cup \cdots \cup V(F_r) = V( F_1 \cdots F_r )$である.
\end{ans}
