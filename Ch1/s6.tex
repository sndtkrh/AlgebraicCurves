\section{Algebraic Subsets of the Plane}

\begin{prob}
  $k=\R$とする.
  (a) $I(V(X^2+Y^2+1)) = (1)$を示せ.
  (b) $\A^2(\R)$の全ての代数的な部分集合は,
  ある$F\in\R[X,Y]$に対して$V(F)$と表されることを示せ.
\end{prob}
\begin{ans}
  (a).
  $V(X^2+Y^2+1) = \emptyset$なので
  $I(V(X^2+Y^2+1)) = I(\emptyset) = (1)$

  (b).
  $V\subset \A^2(\R)$を代数的集合とする.
  $V = \A^2(\R)$のときは$V = V(0)$なのでよい.
  $V \ne \A^2(\R)$とする.
  $V = V_1 \cup \cdots \cup V_r$と既約成分分解する.
  本文中 Corollary 2 より,
  $V_i$は一点か$V(F_i)$である.
  ここで$F_i \in \R[X,Y]$は既約多項式である.
  $V_i \ne \A^2(\R)$のとき,
  各$i$に対して,$V_i = \{ (a,b) \}$のときは
  $F_i = (X-a)^2 + (Y-b)^2 \in \R[X,Y]$とおく.
  すると$V_i = V(F_i)$である.
  したがって,
  $V = V(F_1) \cup \cdots \cup V(F_r) = V( F_1 \cdots F_r )$である.
\end{ans}

\begin{prob}
  (a) $\A^2(\R)$と$\A^2(\C)$での$V(Y^2-XY-X^2Y+X^3)$の既約成分を調べよ.
  (b) $V(Y^2-X(X^2-1))$と$V(X^3+X-X^2Y-Y)$で同様にせよ.
\end{prob}
\begin{ans}
  (a).
  $V(Y^2-XY-X^2Y+X^3)
  = V( (Y-X)(Y-X^2) )
  = V(Y-X) \cup V(Y-X^2)$
  $V(Y-X^2)$が$\A^2(k), k=\C$で既約であることは問題\ref{V(Y-X^2)が既約}で見た.
  これは$k=\R$でも既約である.
  $|V(Y-X^2)| = \infty$なので
  本文中1.6節の系1より
  $Y-X^2 \in \R[X,Y]$が既約多項式であることを示せばよい.
  写像$\phi : \R[X,Y] \to \R[T]; X \mapsto T, Y \mapsto T^2$を考える.
  $\phi(Y-X^2) = T^2 - T^2 = 0$なので
  $\Kernel( \phi ) \supset (Y-X^2)$である.
  逆に$ F \in \Kernel( \phi )$ならば,
  $F = Q(X,Y)(Y-X^2) + R(X), \ R(X) \in \R[X]$と表せるが,
  $0 = \phi(F) = R(T)$である.
  ゆえに$R(X) = 0$である.
  したがって$F \in (Y-X^2)$が分かる.
  よって$\Kernel(\phi) = (Y-X^2)$でこれは素イデアルなので
  $Y-X^2$は既約多項式である.
  よって$V(Y-X^2)$は既約な代数的集合である.
  $V(Y-X)$も同様にして
  $\A^2(k), k=\R,\C$で既約である.
  ゆえに
  \[ V(Y^2-XY-X^2Y+X^3) = V(Y-X) \cup V(Y-X^2) \]
  が求める$\A^2(k), k=\R,\C$上の既約分解である.

  (b).
  $V(Y^2 - X(X^2-1))$は既約である.
  なぜなら$|V(Y^2 - X(X^2-1))| = \infty$であり,
  $Y^2 - X(X^2-1)$は
  問題\ref{楕円曲線は既約}より,
  $k[X,Y], k=\R,\C$上既約であるからである.
  
  $V(X^3+X-X^2Y-Y) = V(X-Y) \cup V(X^2+1)$
  が$\A^2(\R)$における既約分解で,
  $V(X^3+X-X^2Y-Y) = V(X-Y) \cup V(X+\sqrt{-1}) \cup V(X-\sqrt{-1})$
  が$\A^2(\C)$における既約分解である.
\end{ans}
