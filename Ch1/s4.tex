\section{The Hilbert Basis Theorem}

\begin{prob}
  $I$を環$R$のイデアルとし,
  $\pi : R \to R/I$を自然な準同型とする.
  (a) 任意の$R/I$のイデアル$J'$に対して,
  $\pi^{-1}(J') = J$は$I$を含む$R$のイデアルであることを示せ.
  (b) $J$が根基イデアルであるとき,
  またそのときのみ$J'$は根基イデアルであることを示せ.
  (c) $J$が有限生成なら$J'$も有限生成であることを示せ.
  $R$がネーター環ならば$R/I$もネーター環であることを結論付けよ.
  任意の$k[X_1,\dots,X_n]/I$の形の環はネーター環である.
\end{prob}
\begin{ans}
  (a).
  $\pi(I) = 0 \in J' \subset R/I$なので$\pi^{-1}(J') \supset I$である.
  任意の$ a,b \in J$と$r\in R$に対して,
  $\pi(a+b) = \pi(a) + \pi(a) \in J'$なので$a+b \in J$である.
  また$\pi(ra) = \pi(r)\pi(a) \in J'$なので$ra \in J$である.
  よって$J$は$I$を含む$R$のイデアルである.

  (b).
  $J$が根基イデアルであるとする.
  $\bar{a}\in R/I$に対して
  $\bar{a}\in \Rad(J')$ならば$\bar{a} \in J'$であることを示せばよい.
  $\bar{a}\in \Rad(J')$ならば,ある$n\ge 1$が存在して
  $\bar{a}^n = \bar{a^n} \in J'$である.
  よって$a^n \in J$である.
  したがって$a \in \Rad(J) = J$なので
  $\pi(a) = \bar{a} \in J'$である.

  逆に$J'$が根基イデアルであるとする.
  $a^n \in J$ならば$a \in J$であることを示せばよい.
  $a^n \in J$とすると,$\pi(a^n) = \pi(a)^n \in J'$なので
  $\pi(a) \in \Rad(J') = J'$である.
  したがって$a \in J$である.

  (c).
  $J = (a_1, \dots a_n), \ a_i \in R$と有限生成されていると仮定する.
  任意の元$\bar{a} \in J'$をとる.
  $a \in J$なので
  $a = \sum r_i a_i, r_i \in R$と書ける.
  $\pi(a) = \sum \pi(r_i) \pi(a_i) \in J'$である.
  よって任意の$J'$の元は$\pi(a_i)$たちの一次結合でかけるので
  $J' = (\bar{a_1} , \dots , \bar{a_n})$である.

  $R$がネーター環ならば$R$の任意のイデアルは有限生成されているので
  $R/I$の任意のイデアルも有限生成である.
  よって$R/I$はネーター環である.
\end{ans}
