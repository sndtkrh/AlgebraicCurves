\section{Modules; Finiteness Conditions}

\begin{prob}
  $S$が$R$上加群として有限生成ならば$S$は$R$上環として有限生成であることを示せ.
\end{prob}
\begin{ans}
  $S$が$R$上加群として有限生成ならば,
  ある$s_1, \dots , s_n \in S$があって
  任意の$s\in S$に対して
  $r_1, \dots, r_n \in R$が存在し
  $s = r_1 s_1 \cdots r_n s_n $と表される.
  よって任意の$s \in S$は
  $s \in R[s_1, \dots, s_n ]$である.
  したがって$S = R[s_1, \dots, s_n ]$であり,
  $S$は$R$上環として有限生成である.
\end{ans}

\begin{prob}
  $S = R[X]$は環として有限生成であるが
  加群として有限生成でないことを示せ.
\end{prob}
\begin{ans}
  $S = R[X]$は明らかに環として有限生成である.
  しかし,任意の$s_1, \dots s_n \in S = R[X]$に対して
  $ m := \max_{1 \le i \le n} \{ \deg s_i \} $として
  $X^{m+1} \in R[X] = S $を考えると,
  任意の$r_1, \dots , r_n \in R$に対して
  $ \deg ( r_1 s_1 + \cdots + r_n s_n ) < m $
  である.一方,
  $ \deg X^{m+1} = m + 1 $である.
  よって$ r_1 s_1 + \cdots + r_n s_n \ne X^{m+1} $
  である.
  したがって$S$は$R$上加群として有限生成ではない.
\end{ans}
