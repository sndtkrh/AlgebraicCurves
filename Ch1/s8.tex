\section{Modules; Finiteness Conditions}

\begin{prob}
  $S$が$R$上加群として有限生成ならば$S$は$R$上環として有限生成であることを示せ.
\end{prob}
\begin{ans}
  $S$が$R$上加群として有限生成ならば,
  ある$s_1, \dots , s_n \in S$があって
  任意の$s\in S$に対して
  $r_1, \dots, r_n \in R$が存在し
  $s = r_1 s_1 \cdots r_n s_n $と表される.
  よって任意の$s \in S$は
  $s \in R[s_1, \dots, s_n ]$である.
  したがって$S = R[s_1, \dots, s_n ]$であり,
  $S$は$R$上環として有限生成である.
\end{ans}

\begin{prob}
  $S = R[X]$は環として有限生成であるが
  加群として有限生成でないことを示せ.
\end{prob}
\begin{ans}
  $S = R[X]$は明らかに環として有限生成である.
  しかし,任意の$s_1, \dots s_n \in S = R[X]$に対して
  $ m := \max_{1 \le i \le n} \{ \deg s_i \} $として
  $X^{m+1} \in R[X] = S $を考えると,
  任意の$r_1, \dots , r_n \in R$に対して
  $ \deg ( r_1 s_1 + \cdots + r_n s_n ) < m $
  である.一方,
  $ \deg X^{m+1} = m + 1 $である.
  よって$ r_1 s_1 + \cdots + r_n s_n \ne X^{m+1} $
  である.
  したがって$S$は$R$上加群として有限生成ではない.
\end{ans}

\begin{prob}
  $K \subset L$を体とする.
  $L$が$K$上環として有限生成ならば
  $L$は$K$から有限生成された拡大体であることを示せ.
\end{prob}
\begin{ans}
  $L$が$K$上環として有限生成であることから,
  ある$a_1, \dots , a_n \in L$が存在して
  $ L = K[a_1, \dots, a_n] $である.
  よって
  $ L = K( a_1, \dots , a_n ) $である.
  したがって$L$は$K$から有限生成された拡大体である.
\end{ans}

\begin{prob}
  $K \subset L = K(X)$は$K$から有限生成された拡大体であるが,
  $L$は$K$上環として有限生成ではないことを示せ.
\end{prob}
\begin{ans}
  $K \subset L = K(X)$は$K$から有限生成された拡大体である.
  もし$L$が$K$上環として有限生成ならば,
  ある$ s_1, \dots, s_n \in K(X)$があって
  $L = K[s_1, \dots, s_n]$となる.
  $ s_i = \frac{a_i}{b_i} \in K(X), \ a_i, b_i \in K[X]$と表されるので
  $b = b_1 \cdots b_n \in K[X]$とおくと,
  任意の$z \in L$に対して,ある$n$が存在して
  $ b^n z \in K[x]$となるが
  問題 \ref{} より$K[X]$の既約多項式は無限に存在するので
  $b$の既約分解に現れないような既約多項式$c \in K[X]$をとることができる.
  この$c$で$z := \frac{1}{c}$とおくと
  ある$n$が存在して$ \frac{b^n}{c} \in K[x] $となる.
  つまり$c$は$b$を割り切ることになるがこれは$c$の取り方に矛盾する.
  よって$L$は$K$上環として有限生成ではない.
\end{ans}

\begin{prob}
  $R$を$S$の部分環とし,$S$を$T$の部分環とする:$R \subset R \subset T$.
  \begin{itemize}
  \item[(a)] $S = \sum_i Rv_i, T = \sum_j S w_j$ならば$ T = \sum_{i,j} R v_i w_j $を示せ.
  \item[(b)] $S = R{v_1, \dots v_n], T = S[w_1, \dots ,w_m]$ならば$T = R[v_1, \dots v_n, w_1, \dots ,w_m]$を示せ.
  \item[(c)] $R,S,T$が体で$S = R(v_1,\dots,v_n), T = S(w_1, \dots, w_m )$ならば$ T = R(v_1,\dots,v_n, w_1, \dots, w_m )$を示せ.
  \end{itemize}
  よって,3つの有限性の条件はそれぞれ推移的関係である.
\end{prob}
\begin{ans}
  (a).
  $S = \sum_{i=1}^n Rv_i, T = \sum_{j=1}^m S w_j$とする.
  任意の$t \in T$に対して
  ある$s_1, \dots , s_m \in S$があって
  $ t = s_1 w_1 + \cdots + s_m w_m $となる.
  各$s_j$について
  ある$ r_{j1}, \dots r_{jn} \in R $があって
  $ s_j = r_{j1} v_1 + \cdots r_{jn} v_n $である.
  よって
  $ t = s_1 w_1 + \cdots + s_m w_m
  = (r_{11} v_1 + \cdots + r_{1n} v_n ) w_1 + \cdots + (r_{m1} v_1 + \cdots + r_{mn} v_n ) w_m
  = \sum_{i,j} r_{ji} v_i w_j \in \sum_{ij} R v_i w_j $である.
  ゆえに$ T = \sum_{i,j} R v_i w_j $である.
  
  (b).
  $S = R{v_1, \dots v_n], T = S[w_1, \dots ,w_m]$ならば
  $T = R[v_1, \dots v_n] [w_1, \dots ,w_m]
  =  R[v_1, \dots v_n, w_1, \dots ,w_m]$

  (c).
  $R,S,T$が体で
  $S = R(v_1,\dots,v_n), T = S(w_1, \dots, w_m )$とする.
  $ T = R(v_1,\dots,v_n) (w_1, \dots, w_m )$なので
  任意の$F \in T $に対して
  $F = \frac{F_1}{F_2}, \ F_1,F_2 \in R(v_1,\dots,v_n)[w_1, \dots, w_m]$と表される.
  ある$G_i \in R[v_1, \dots, v_n] $があって
  $ G_i F_i \in R[v_1,\dots,v_n, w_1, \dots, w_m]$となる.
  よって$F = \frac{F_1}{F_2} = \frac{ G_1 G_2 F_1 }{ G_1 G_2 F_2 } \in R(v_1,\dots,v_n, w_1, \dots, w_m )$である.
  したがって$ T =  R(v_1,\dots,v_n, w_1, \dots, w_m )$である.
\end{ans}
