\section{Algebraic Preliminaries}
\begin{prob}
  $R$を整域とする.
  (a) $F,G\in R[X_1, \dots ,X_n]$が次数$r,s$の斉次式であるとき,
  $FG$が次数$r+s$の斉次式であることを示せ.
  (b) $R[X_1, \dots ,X_n]$の任意の斉次式の因子はまた斉次式であることを示せ.
\end{prob}
\begin{ans}
  (a).
  $F,G$は$R[X_1, \dots ,X_n]$の斉次式なので
  $F = \sum_i a_{(i)} X^{(i)}, G = \sum_j b_{(j)} Y^{(j)}$と書ける.
  ここで$X^{(i)}, Y^{(j)}$は次数$r,s$の単項式である.
  $FG = ( \sum_i a_{(i)} X^{(i)} ) ( \sum_j a_{(j)} X^{(j)} )
  = \sum_{i,j} a_{(i)} b_{(j)} X^{(i)} Y^{(j)}$であり,
  $R$は整域なので各$a_{(i)} b_{(j)} X^{(i)}Y^{(j)}$は次数$r+s$の単項式である.
  よって$FG$は次数$r+s$の斉次式である.

  (b).斉次式$F \in R[X_1, \dots ,X_n]$が
  $F = GH, G = \sum_{i} a_{(i)} X^{(i)}, H = \sum_{j} b_{(j)} Y^{(j)}$
  と分解されたとする.
  $G$が斉次式でないと仮定する.
  すると$G$の項のうち,
  最も次数が小さいもの$a_{(s)}X^{(s)}$と
  最も次数が大きいもの$a_{(t)}X^{(t)}$があって
  $\deg(a_{(s)}X^{(s)}) < \deg(a_{(t)}X^{(t)})$が成り立つ.
  また,$H$の項のうち,
  最も次数が小さいものを$b_{(p)}Y^{(p)}$
  最も次数が大きいものを$b_{(q)}Y^{(q)}$
  とすると
  $\deg(b_{(p)}Y^{(p)}) \le \deg(b_{(q)}Y^{(q)})$が成り立つ.
  $F=GH$の項で
  $X^{(s)}Y^{(p)}$の係数は$0$でなく
  $X^{(t)}Y^{(q)}$の係数は$0$でない.
  $s,t,p,q$の取り方からこの2つの単項式の次数は異なる.
  よって$F = GH$が斉次式であることに矛盾する.
  したがって$G$は斉次式である.
\end{ans}

\begin{prob}
  $R$をUFDとし,$K$を$R$の商体とする.
  $K$の任意の元$z$は$z=a/b$と書けることを示せ.
  ここで$a,b$は共通因子を持たない$R$の元である.
\end{prob}
\begin{ans}
  $\forall z \in K$をとり,
  $z = a/b, \ a,b \in R$と書く.
  $a,b$が共通因子を持たなければ,よい.
  $a,b$が共通因子を持つとする.
  $R$はUFDなので
  $d = \mathrm{GCD}(a,b)$とすると
  $a = da_1, b = db_1$と書けて
  $a_1$と$b_1$は互いに素である.
  このとき$a_1/b_1 = a/b = z$であるので
  題意は示された.
\end{ans}

\begin{prob}
  $R$をPIDとする.
  $P$を$R$の零でない真の素イデアルとする.
  (a) $P$は既約元で生成されていることを示せ.
  (b) $P$は極大であることを示せ.
\end{prob}
\begin{ans}
  (a).
  $R$はPIDなので$P = (p), \exists p \in R$と書ける.
  もし$p$が既約元でない,すなわち
  $p = ab, \ a,b\in R$と分解されて$a,b$は単元でないとする.
  このとき$a,b \not\in P$であるが$ab \in P$である.
  これは$P$が素イデアルであることに反する.
  よって$p$は既約元である.

  (b).
  $I$を$R$のイデアルで,$P$を真に含むとする.
  $I = (a), \exists a\in R$と書け,
  $p \in (p) = P \subsetneq I = (a)$
  なので$p = ba, \exists b \in R$となる.
  $p$は既約元なので$a$は単元か単元と$p$の積でなければならず,
  しかも$P \subsetneq I$であることから
  $a$は単元でなければならない.
  ゆえに$I = (a) = R$となる.
  したがって$P$は極大である.
\end{ans}
