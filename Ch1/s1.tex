\section{Algebraic Preliminaries}
\begin{prob}
  $R$を整域とする.
  (a) $F,G\in R[X_1, \dots ,X_n]$が次数$r,s$の斉次式であるとき,
  $FG$が次数$r+s$の斉次式であることを示せ.
  (b) $R[X_1, \dots ,X_n]$の任意の斉次式の因子はまた斉次式であることを示せ.
\end{prob}
\begin{ans}
  (a).
  $F,G$は$R[X_1, \dots ,X_n]$の斉次式なので
  $F = \sum_i a_{(i)} X^{(i)}, G = \sum_j b_{(j)} Y^{(j)}$と書ける.
  ここで$X^{(i)}, Y^{(j)}$は次数$r,s$の単項式である.
  $FG = ( \sum_i a_{(i)} X^{(i)} ) ( \sum_j a_{(j)} X^{(j)} )
  = \sum_{i,j} a_{(i)} b_{(j)} X^{(i)} Y^{(j)}$であり,
  $R$は整域なので各$a_{(i)} b_{(j)} X^{(i)}Y^{(j)}$は次数$r+s$の単項式である.
  よって$FG$は次数$r+s$の斉次式である.

  (b).斉次式$F \in R[X_1, \dots ,X_n]$が
  $F = GH, G = \sum_{i} a_{(i)} X^{(i)}, H = \sum_{j} b_{(j)} Y^{(j)}$
  と分解されたとする.
  $G$が斉次式でないと仮定する.
  すると$G$の項のうち,
  最も次数が小さいもの$a_{(s)}X^{(s)}$と
  最も次数が大きいもの$a_{(t)}X^{(t)}$があって
  $\deg(a_{(s)}X^{(s)}) < \deg(a_{(t)}X^{(t)})$が成り立つ.
  また,$H$の項のうち,
  最も次数が小さいものを$b_{(p)}Y^{(p)}$
  最も次数が大きいものを$b_{(q)}Y^{(q)}$
  とすると
  $\deg(b_{(p)}Y^{(p)}) \le \deg(b_{(q)}Y^{(q)})$が成り立つ.
  $F=GH$の項で
  $X^{(s)}Y^{(p)}$の係数は$0$でなく
  $X^{(t)}Y^{(q)}$の係数は$0$でない.
  $s,t,p,q$の取り方からこの2つの単項式の次数は異なる.
  よって$F = GH$が斉次式であることに矛盾する.
  したがって$G$は斉次式である.
\end{ans}

\begin{prob}
  $R$をUFDとし,$K$を$R$の商体とする.
  $K$の任意の元$z$は$z=a/b$と書けることを示せ.
  ここで$a,b$は共通因子を持たない$R$の元である.
\end{prob}
\begin{ans}
  $\forall z \in K$をとり,
  $z = a/b, \ a,b \in R$と書く.
  $a,b$が共通因子を持たなければ,よい.
  $a,b$が共通因子を持つとする.
  $R$はUFDなので
  $d = \mathrm{GCD}(a,b)$とすると
  $a = da_1, b = db_1$と書けて
  $a_1$と$b_1$は互いに素である.
  このとき$a_1/b_1 = a/b = z$であるので
  題意は示された.
\end{ans}

\begin{prob}
  $R$をPIDとする.
  $P$を$R$の零でない真の素イデアルとする.
  (a) $P$は既約元で生成されていることを示せ.
  (b) $P$は極大であることを示せ.
\end{prob}
\begin{ans}
  (a).
  $R$はPIDなので$P = (p), \exists p \in R$と書ける.
  もし$p$が既約元でない,すなわち
  $p = ab, \ a,b\in R$と分解されて$a,b$は単元でないとする.
  このとき$a,b \not\in P$であるが$ab \in P$である.
  これは$P$が素イデアルであることに反する.
  よって$p$は既約元である.

  (b).
  $I$を$R$のイデアルで,$P$を真に含むとする.
  $I = (a), \exists a\in R$と書け,
  $p \in (p) = P \subsetneq I = (a)$
  なので$p = ba, \exists b \in R$となる.
  $p$は既約元なので$a$は単元か単元と$p$の積でなければならず,
  しかも$P \subsetneq I$であることから
  $a$は単元でなければならない.
  ゆえに$I = (a) = R$となる.
  したがって$P$は極大である.
\end{ans}

\begin{prob}\label{全ての点に対して0なら零多項式}
  $k$を無限体とし,
  $F \in k[X_1, \dots ,X_n]$であるとする.
  任意の$a_1,\dots ,a_n \in k$に対して
  $f(a_1,\dots,a_n) = 0$ならば$F = 0$であることを示せ.
\end{prob}
\begin{ans}
  $n$に関する数学的帰納法で証明する.
  $n=1$のとき,
  $F(X) = \sum_i c_i X^i \in k[X]$と書ける.
  $F$の$k$における根は高々$\deg(F) < \infty$個なので
  任意の$a \in k$について$F(a) = 0$となるのは$F = 0$のときしかない.
  $n$より小さいときに成り立つとして$n$のとき,
  $ F = \sum_i F_i X_n^i \in k[X_1,\dots ,X_n] = k[X_1,\dots,X_{n-1}][X_n]$と書ける.
  任意の$a_1,\dots,a_{n-1} \in k$に対して,
  $F(a_1, \dots, a_{n-1}, X_n) = \sum_i F_i(a_1,\dots,a_{n-1}) X_n \in k[X_n]$の根は
  有限個しかない.
  よって任意の$a_1,\dots, a_{n-1}, a_n \in k$に対して
  $F(a_1, \dots, a_{n-1}, a_n) = 0$となるには
  各$i$について$F_i (a_1, \dots, a_{n-1}) = 0$でなければならない.
  帰納法の仮定から$F_i = 0 \ \forall i$である.
  したがって$F = 0$である.
\end{ans}

\begin{prob}\label{無限個のモニック既約多項式}
  $k$を体とする.$k[X]$には無限個のモニック既約多項式が存在することを示せ.
\end{prob}
\begin{ans}
  モニック既約多項式が有限個しかないと仮定して
  $F_1,\dots,F_n \in k[X]$を全てのモニック既約多項式とする.
  このとき$F = F_1 \cdots F_n + 1 \in k[X]$はモニックである.
  一方$k[X]$はUFDなので
  $F$は既約元の積に一意的に分解される.
  $F$の作り方から$F$自身が既約元になるしかない.
  ゆえに$F$は$F_1,\dots,F_n$とは異なるモニック既約多項式となるが
  これは仮定に反する.
  したがって$k[X]$には無限個のモニック既約多項式が存在する.
\end{ans}

\begin{prob}\label{代数的閉体は無限集合}
  任意の代数的閉体は無限個の元を持つ.
\end{prob}
\begin{ans}
  $k$を代数的閉体とする.
  問題 \ref{無限個のモニック既約多項式} より
  $k[X]$のモニック既約多項式は無限個存在する.
  代数的閉体$k$上のモニック既約多項式は$X-a \in k[X], \ a\in k$という形をしているので
  $k$は無限個の元を持たなければならない.
\end{ans}

\begin{prob}
  $k$を体として$F\in k[X_1,\dots,X_n], \ a_1,\dots,a_n \in k$とする.
  (a) ある$\lambda_(i) \in k$たちが存在して,
  \[ F = \sum \lambda_{(i)} (X_1 - a_1)^{i_1} \cdots (X_n - a_n)^{i_n} \]
  となることを示せ.
  (b) もし$F(a_1,\dots,a_n) = 0$ならば
  $G_i \in k[X_1,\dots,X_n]$が存在して
  $F = \sum_i^{n} (X_i - a_i) G_i$となることを示せ.
\end{prob}
\begin{ans}
  (a).
  $n$に関する数学的帰納法で証明する.

  $n=1$のとき,$F = \sum_{i=1}^d c_i X_1^i \in k[X_1]$
  と書ける.
  次数$d$に関する帰納法で$F = \sum_{i=1}^d \lambda_{(i)}(X_1 - a_1)^i$の形に書けることを証明する.
  $d=0$のときは自明.
  次数が$d$未満については主張が成り立つとして次数が$d$のとき,
  最高次の項$c_d X_1^d$を多項式$X_1 - a_1$で割って,
  $c_d X_1^d = c_d (X_1 - a_1)^d + R(x)$と表すことができる.
  ここで$R(x) \in k[X_1]$は高々$d-1$次の多項式である.
  すると,$F = c_d (X_1 - a_1)^d + R(x) + \sum_{i=1}^{d-1} c_i X_1 $となる.
  $R(x) + \sum_{i=1}^{d-1} c_i X_1 $は$d-1$次式なので仮定より,
  $\sum_{i=1}^{d-1} \lambda_{(i)} (X_1 - a_1)^i$と表すことができる.
  ゆえに$\lambda_{(d)} = c_d$とおけば
  $F = \sum_{i=0}^d \lambda_{(i)} (X_1 - a_1)^i$と表すことができる.

  $n$未満については主張が成り立つとして$n$のとき,
  $F$を文字$X_n$について整理して,
  $F = \sum_{i=1}^d G_i X_n^i $と書くことができる.
  ここで$G_i \in k[X_1,\dots,X_{n-1}]$である.
  $X_n$について最高次の項$G_d X^d$を考える.
  $G_d X_n^d = G_d (X_n - a_n)^d + G_d R(x)$と表すことができる.
  ここで$G_d R(x) \in k[X_1,\dots,X_n]$は$X_n$について高々$d-1$次の多項式である.
  帰納法の仮定より
  $G_d = \sum_i \lambda_{(i)} (X_1 - a_1)^{i_1} \cdots (X_{n-1} - a_{n-1})^{i_n}$
  と書けるので$G_d (X_n - a_n)^d$は
  $\sum_i \lambda_{(i)} (X_1 - a_1)^{i_1} \cdots (X_{n-1} - a_{n-1})^{i_n} (X_n - a_n)^d$
  という形をしている.
  $X_d$について$d-1$次以下の部分も次数の大きい方から処理していけば
  主張が示されることが分かる.

  (b).
  $F(a_1,\dots,a_n) = 0$なら,
  $F = \sum \lambda_{(i)} (X_1 - a_1)^{i_1} \cdots (X_n - a_n)^{i_n}, \ \lambda_{(i)} \in k $
  と書けるので,
  $F$の各項についてある$j$が存在して$i_j \ge 1$でなければならない.
  よってそのような$j$について$(X_j - a_j)$でくくると,
  $G_i \in k[X_1,\dots,X_n]$が存在して
  $F = \sum_i^{n} (X_i - a_i) G_i$となる.
\end{ans}
