\section{The Ideal of a Set of Points}

\begin{prob}
  $V,W \subset \A^n(k)$を代数的集合とする.
  $V = W$であるのは$I(V) = I(W)$のときであり,
  またこのときに限ることを示せ.
\end{prob}
\begin{ans}
  一般に$X,Y$を$\A^n(k)$の代数的集合であるとする.
  $X\subset Y$のとき$I(X) \subset I(Y)$である.
  また,$X = V(I(X)), Y = V(I(Y))$である.  
  よって$V = W \Rightarrow I(V) = I(W)$かつ
  $I(V) = I(W) \Rightarrow V(I(V)) = V(I(W)) \Rightarrow V = W$である.
\end{ans}

\begin{prob}
  (a) $V \subset \A^n(k)$を代数的集合とする.
  $P \in \A^n(k)$を$P\not\in V$である点とする.
  このとき,
  多項式$F \in k[X_1, \dots X_n] $であって,
  全ての$Q\in V$について$F(Q) = 0$となり,
  かつ$F(P) = 1$であるようなものが存在することを示せ.
  (b) $P_1,\dots, P_r \in \A^n(k) \setminus V$を異なる点とする.
  このとき多項式$F_1, \dots, F_n \in I(V)$であって,
  $i\ne j$に対して$F_i(P_j) = 0$であり,
  かつ$F_i(P_i) = 1$であるようなものが存在することを示せ.
  (c) $P_1,\dots, P_r$,$V$を(b)で定めたものとする.
  また$a_{ij} \in k, \ 1\le i,j \le r$とする.
  このとき$G_i \in I(V)$であって
  任意の$i,j$に対して$G_i(P_j) = a_{ij}$を満たすことを示せ.
\end{prob}
\begin{ans}
  (a).
  $V$は代数的集合であるから
  $V = V(S), S \subset k[X_1,\dots,X_n]$と書ける.
  $G \in S$で$G(P) = a \ne 0$である多項式をとることができる.
  なぜなら,もし$\forall G \in S, G(P) = 0$なら$P\in V$となってしまうからである.
  $V$の定義から任意の$Q\in V$に対して$G(Q) = 0$である.
  $F = \frac{1}{a} G$を$G$のすべての係数を$\frac{1}{a}$倍した多項式とする.
  このとき任意の$Q \in V$に対して$F(Q) = 0$であり,
  かつ$F(P) = \frac{1}{a}G(P) = \frac{1}{a} a = 1$である.
  よってこの$F$が求める多項式である.
  また$F \in I(V)$である.

  (b).
  各$i=1,\dots, r$に対して,
  $I( V \cup \{ P_1 , \dots , P_n \} ) \subsetneq I( V \cup \{ P_1 , \dots , \hat{P_i}, \dots , P_n \})$
  であり,$W_i := V \cup \{ P_1 , \dots , \hat{P_i}, \dots , P_n \}$は代数的集合なので
  (a)の結果から$F_i \in I(W_i) \subset I(V)$であって,
  任意の$Q\in W_i$に対して$F_i(Q) = 0$,
  特に$j \ne i$に対して$F_i(P_j) = 0$であって
  $F_i(P_i) = 1$であるようなものが存在する.
  この$F_1, \dots, F_r$たちが求める多項式である.
  
  (c).
  (b)で構成した$F_1, \dots, F_r$をとる.
  各$i$について$G_i := a_{i1} F_1 + \cdots + a_{ir} F_r$と定めればよい.
\end{ans}

\begin{prob}\label{素イデアルならば根基イデアル}
  $I\subset R$をイデアルとする.
  $a^n \in I$かつ$b^m\in I$ならば
  $(a+b)^{n+m} \in I$を示せ.
  $\Rad(I)$はイデアルであることを示せ.
  任意の素イデアルは根基イデアルであることを示せ.
\end{prob}
\begin{ans}
  $a^n \in I$かつ$b^m\in I$とする.
  $(a+b)^{n+m} = \sum_{k} \binom{n+m}{k} a^{n+m-k}b^k$
  である.
  各$k$について,$n+m-k \ge n$または$k \ge m$が成り立つ.
  よって$(a+b)^{n+m} \in I$である.

  $\Rad(I)$がイデアルであることを示す.
  まず,$\Rad(I) \supset I \ne \emptyset $である.
  $\Rad(I)$は上で示したことから和について閉じている.
  さらに任意の$a \in \Rad(I), \ r\in R$に対して,
  ある$n$が存在して$ a^n \in I$が成り立つので,
  $ (ra)^n = r^n a^n \in I $である.
  したがって$ra \in \Rad(I)$が成り立つ.
  よって$\Rad(I)$はイデアルである.

  $P\subset R$が素イデアルであるとする.
  $P \subset \Rad(P)$は自明.
  $P \supset \Rad(P)$を示す.
  $ a\in \Rad(P)$をとる.
  ある$n$が存在して
  $ a^n \in P$が成り立つ.
  $P$は素イデアルであるので
  $a\in P$または$a^{n-1} \in P$が成り立つ.
  $a\in P$ならばよい.
  $a^{n-1} \in P$なら帰納的に$a\in P$が言える.
  よって$P \supset \Rad(P)$である.
\end{ans}

\begin{prob}
  $I = (X^2 + 1) \subset \R[X]$が根基イデアルである(素イデアルでさえある)が,
  $I$は$\A^1(\R)$の集合のイデアルとはならないことを示せ.
\end{prob}
\begin{ans}
  $X^2 + 1$は$\R[X]$の既約多項式である.
  $\R[X]$はUFDなので$X^2+1$は素元であり,
  ゆえに$(X^2+1) \subset \R[X]$は素イデアルである.
  特に問題\ref{素イデアルならば根基イデアル}より
  根基イデアルである.
  $F(X) = X^2 + 1$とする.
  任意の$a \in \R$について$F(a) \ne 0$である.
  よって$(F)$は$\A^1(\R)$の集合のイデアルとはなり得ない.
\end{ans}

\begin{prob}
  任意のイデアル$I \subset k[X_1,\dots,X_n]$について,
  $V(I) = V(\Rad(I))$であることと,
  $\Rad(I) \subset I(V(I))$であることを示せ.
\end{prob}
\begin{ans}
  $I \subset \Rad(I)$なので
  $V(I) \supset V(\Rad(I))$は自明.
  $a = (a_1, \dots , a_n) \in V(I)$を任意にとる.
  任意の$F\in I$について$F(a) = 0$である.
  $G \in V(\Rad(I))$を任意の元とする.
  ある$n$が存在して$G^n \in I$なので
  $G^n(a) = 0$である.
  体$k$は整域なので$G(a) = 0$である.
  よって$a \in V(\Rad(I))$である.
  つまり$V(I) \subset V(\Rad(I))$である.

  $\Rad(I) \subset I(V(I))$であること示す.
  上で示したことから$V(I) = V(\Rad(I))$なので
  $I(V(I)) = I(V(\Rad(I)))$である.
  よって本文中の性質(8)より
  $\Rad(I) \subset I(V(\Rad(I))) = I(V(I))$である.
\end{ans}
